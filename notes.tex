\documentclass[notes]{agony}
\newcommand{\xgets}{\xleftarrow}
\newcommand{\one}{\symbb{1}}
\newcommand{\ord}{\operatorname{ord}}
\newcommand{\leg}[2]{\qty(\frac{#1}{#2})}
\newcommand{\mgrp}[1]{(\Z/#1\Z)^\times}

\title{CO 485/685 Fall 2022: Lecture Notes}
\begin{document}
\renewcommand{\contentsname}{CO 485/685 Fall 2022:\\{\huge Lecture Notes}}
\thispagestyle{firstpage}
\tableofcontents

Lecture notes taken, unless otherwise specified,
by myself during the Fall 2022 offering of CO 485/685,
taught by David Jao.

Chapter/lecture titles are made-up nonsense and do not follow the textbook
or any other published resource.
Actually, scratch that, this entire document is nonsense because
I am literally auditing this course two nested prerequisites behind.

\chapter{Introduction to Cryptography}

\section{(09/07; skipped)}

\begin{markdown}
## Almost-Public Key Cryptosystems (09/09)

- For a symmetric key cryptosystem, require sets of key space $K$, message space $M$, and ciphertext space $C$
    - Define encryption function $Enc : K \to M \to C$ and decryption $Dec : K \to C \to M$
    - Correctness property: for all $k$, $Dec(k)$ is a left inverse of $Enc(k)$
    - Symmetric means that both decryption and encryption use shared secret $k$, which we assume is drawn randomly from $K$
- Public key encryption scheme (Diffie, Hellman, Merkle, c. 1976)
    - Setup similar: message space $M$ and ciphertext space $C$ but with two key spaces $K_1$ of public keys and $K_2$ of private keys
    - Define $Enc : K_1 \to M \to C$ and $Dec : K_2 \to C \to M$
    - Define $KeyGen : \mathbb{1}^\ell \to R \subset K_1 \times K_2$
        - For some reason, let $\mathbb{1}^n$ be the unary representation of $n$??
    - Correctness: for all $(k_1,k_2) \in R$ related, $Dec(k_2)$ is a left inverse of $Enc(k_1)$
- Merkle puzzle (1974) 
    - Each party creates ``puzzle'' which is hard to solve but not too hard
    - Alice generates 1,000,000 puzzles and sends them to Bob
    - Bob solves one of the puzzles arbitrarily and sends half of the answer to Alice
    - Alice knows the answer, so Alice knows the second half of the answer, which becomes the shared secret
    - Eve cannot (realistically) solve 500,000 puzzles in time to intercept
- Diffie--Hellman key exchange
    - Consider the multiplicative group $G = (\Z / p\Z)^* = \{1,\dotsc,p-1\}$ and some arbitrary element $g \in G$ with sufficiently large order
    - Alice privately picks some $x \in \Z$, computes $g^x$, and sends it to Bob
    - Bob privately picks some $y \in \Z$, computes $g^y$, and sends it to Alice
    - Both can now calculate a shared secret $k = g^{xy} = (g^x)^y = (g^y)^x$
    - Eve would have to solve the Diffie--Hellman problem: given $p$, $g$, $g^x$, $g^y$, find $g^{xy}$ which is known to be hard
- Clifford Cocks privately discovered RSA 1973, DH 1974 for GCHQ (if you believe the intelligence community)

## A Public Key Cryptosystem -- RSA (09/12)

- RSA (Rivest, Shamir, Adleman 1977): first cryptosystem and remains secure
- Theoretically secure, but implementations are ass (cf. ``Fuck RSA'')
- MATH 135 review of the algorithm:
    - This ``textbook RSA'' has practical flaws and is insecure
    - $KeyGen : \mathbb{1}^\ell \to (pk, sk) \in R$
        1. Choose random primes $p,q \approx 2^\ell$ where $p$ and $q$ are odd and distinct
        2. Compute $n = pq$
        3. Choose $e \in (\Z / \phi(n)\Z)^\times$ where $\phi(n) = (p-1)(q-1)$
        4. Compute $d = e^{-1} \bmod \phi(n)$
        5. Disclose public key $(n,e)$ and keep secret key $(n,d)$
    - $Enc : K_1 \to M \to C : (n,e) \mapsto m \mapsto m^e \bmod n$ where $M = (\Z/n\Z)^\times = \{ x : \Z / n\Z : \gcd(x,n) = 1\} = C$
        - Weird that $M$ depends on $n$ (part of the key). In practice, it doesn't matter because the only messages that divide $n$ are the primes, which breaks RSA anyways
    - $Dec : K_2 \to C \to M : (n,d) \mapsto c \mapsto c^d \bmod m$
- Correctness: Must show that $(m^e \bmod n)^d \bmod n = m$
    *Proof*. $(m^e \bmod n)^d \bmod n = m^{ed} \bmod n$ (exponentiation under mod). Then, since $d = e^{-1} \bmod \phi(n)$, there exists $k$ such that $de - 1 = k\phi(n)$, we have $m^{\phi(n)k + 1} \equiv (m^{\phi(n)})^km \equiv m \pmod{m}$. This holds by Euler's theorem ($\forall m \in (\Z/n\Z)^\times, m^{\phi(n)} \equiv 1 \pmod n$) or Fermat's Little Theorem + Chinese Remainder Theorem (MATH 135)
- Security: Trivial that factoring $n=pq$ breaks RSA by computing $\phi(n)$
    - Conversely, if you know $\phi(n) = (p-1)(q-1)$ you can take $q\phi(n) = (n-1)(q-1)$ and solve for $q$
        - To avoid this, use the Carmichael exponent $\lambda(n) = \lcm(p-1,q-1)$ instead of $\phi(n)$ which works. Of course, this doesn't work in practice because it's not actually that much different
    - For any non-trivial case, knowing one pair $(e,d)$ also allows factoring $n$
    - Must make an assumption about hardness to prove security:
        - Factoring assumption: factoring random integers is hard
        - RSA factoring assumption: factoring $n=pq$ is hard (see, e.g., elliptical curve algorithm which depends on size of smallest prime in the factorization)
            - Of course, quantum computing fucks all of this to hell (see troll PQRSA which uses many small primes to make terabyte-sized moduli)
        - RSA assumption: given $n$, $e$, $m^e\bmod n$, it is hard to find $m$
    - Can prove RSA assumption $\implies$ RSA works (cannot prove without assumption without better results from complexity theory)

## Security Definitions (09/14)

- Security definitions, e.g., OW-CPA, IND-CPA, IND-CCA (Boneh, Shoup)
- How secure is a cryptosystem? Specify:
    - Allowable interactions between adversaries and parties
        - Second part of abbreviation
    - Computational limits of adversary
        - Not usually specified, usually probabilistic polynomial time
    - Goal of the adversary to ``break'' the cryptosystem
        - First part of abbreviation
- OW-CPA: ``one-way chosen-plaintext attack''
    - Adversary, given public key $pk$ and encryption $c$ of message $m$ under $pk$, wants to determine $m$
    - Formally, given a random $pk$ and $c$ such that $c = Enc(pk, m)$ for some random $m$, it is infeasible for any probabilistic polynomial time algorithm $\mathcal A$ to determine $m$ with non-negligible probability. That is, $\Pr[\mathcal A(pk,c) = m] = O(\frac{1}{\lambda^c})$ for all $c > 0$.
- Easier way to formalize (``Sequences of Games'', Shoup 2004)
    - Two players: challenger $\mathcal C$ and adversary $\mathcal A$
    - Then, OW-CPA is
        1. $\mathcal C$ runs $KeyGen : \one^\lambda \xto{\$} (pk, sk)$
        2. $\mathcal C$ chooses $m \xgets{\$} M$
        3. $\mathcal C$ computes $c \gets Enc(pk, m)$
        4. $m' \xgets{\$} \mathcal A(pk, c)$
        - with the win condition that $m' = m$, and we say that a cryptosystem is OW-CPA if a probabilistic polynomial time adversary $\mathcal A$ cannot win this game with non-negligible probability
    - IND-CPA (Goldmeier, Micoli 1984): indistinguishability
        1. $\mathcal C$ runs $(pk, sk) \xgets{\$} KeyGen(\one^\lambda)$
        2. $(m_0,m_1) \xgets{\$} \mathcal A(\one^\lambda, pk)$
        3. $\mathcal C$ picks $b \xgets{\$} \{0,1\}$
        4. $\mathcal C$ computes $c \xgets{\$} Enc(pk, m_b)$
        5. $b' \xgets{\$} \mathcal A(\one^\lambda, pk, c)$
        - with the win condition $b = b'$, and a cryptosystem is IND-CPA if for all prob.\ poly.\ time $\mathcal A$, $\abs{\frac12 - \Pr[\text{win}]} = O(\frac{1}{\lambda^\varepsilon})$ for all $\varepsilon > 0$
        - Encryption function must be random, otherwise $\mathcal A$ can re-encrypt

## Actual IND-CPA systems (09/16)

- IND-CPA is the standard security definition for symmetric security
    - Ciphertext contains no information about plaintext (except length)
- Design a slightly different equivalent IND-CPA game:
    1. $\mathcal C$ runs $(pk, sk) \xgets{\$} KeyGen(\one^\lambda)$
    2. $(m_0,m_1) \xgets{\$} \mathcal A(\one^\lambda, pk)$
    3. $\mathcal C$ picks $b \xgets{\$} \{0,1\}$
    4. $\mathcal C$ computes $c_1 \xgets{\$} Enc(pk, m_b)$ and $c_2 \xgets{\$} Enc(pk, m_{b-1})$
    5. $b' \xgets{\$} \mathcal A(\one^\lambda, pk, c_1, c_2)$
- Consider textbook RSA: $\mathcal A$ can choose $m_0 \neq m_1$ and compute $Enc(pk, m_0)$ and $Enc(pk, m_1)$ which allows it to win
    - In general, this applies to any scheme with deterministic encryption
- Goldwasser-Micali (``Probabilistic Encryption'' 1982)
    1. Pick $n = pq$ (useful to have $p \equiv q \equiv 3 \pmod 4$)
    2. Pick $r \in (\Z/n\Z)^\times$ such that $r \not\equiv x^2 \pmod p$ and $r \not\equiv x^2 \pmod q$
    3. Define $pk = (n, r)$ and $sk = (p, q)$
    4. Select a message bit $b$ from $M = \{0,1\}$
    5. Encrypt $Enc(b) = r^b y^2$ for some $y \xgets{\$}(\Z/n\Z)^\times$
    - Then, decrypt by determining ciphertext's squareness mod $n$
        - This is easy with the factorization $n=pq$ by Euler's criterion ($a$ is square mod prime $p$ if and only if $a^{(p-1)/2} \equiv 1 \pmod p$)
        - Determining squareness without factorization of $n$ is hard, apparently
    - Since plaintexts are one bit, OW $\iff$ IND and this is provable under the circular-y assumption that determining squareness is hard
    - Also one bit messages are literally useless so who cares
- Elgamal (1984) (sometimes IND-CPA)
    - Publickeycryptosystemified Diffie-Hellman
    1. Setup is the same as DH, take some element $g \in G$ of a group
    2. Define $pk = g^x$ and $sk = x$
    3. Encrypt $Enc(m) = (g^y, g^{xy}\cdot m)$ for $y \xgets{\$} \Z$
    - Then, decrypt $Dec(c_1, c_2) = \frac{c^2}{c_1^x} = \frac{g^{xy}\cdot m}{(g^y)^x} = m$
    - In general, key sharing schemes can be cryptosystemified like this
    - In an IND-CPA game, given $(g^y, g^{xy}m_b)$
        - Divide out $m_0$ to get either $g^{xy}$ (if $m_b = m_0$) or garbage
        - Real challenge is distinguishing $g^{xy}$ from garbage
    - Decisional Diffie-Hellman assumption: in the following game, $\abs{\Pr[\mathcal A\text{ wins}]-\frac12}$ is negligible in $\lambda$
        1. $\mathcal C$ chooses $p \xgets{\$} \Z$ prime, $p \approx 2^\lambda$
        2. $\mathcal C$ chooses $g \in (\Z/p\Z)^\times$
        3. $\mathcal C$ chooses $x,y \xgets{\$} \Z$ and $h \xgets{\$} (\Z/p\Z)^\times$, computes $g_1 = g^x$, $g_2 = g^y$, $g_3 = g^{xy}$
        4. $\mathcal C$ chooses $b \xgets{\$} \{0,1\}$ and $g_4 = g_3$ if $b=0$ and $h$ if $b=1$
        5. $b' \gets \mathcal A(\one^\lambda,p,g,g_1,g_2,g_4)$
    - Can prove: if DDH assumption holds, Elgamal is IND-CPA
- Layers of assumptions here:
    - DLOG: given $g$ and $g^x$, it is hard to find $x$
    - CDH: given $g$, $g^x$, and $g^y$, it is hard to find $g^{xy}$ (equivalent to Elgamal being OW-CPA)
    - DDH: given $g^{xy}$ and garbage, is hard to distinguish the garbage
- How to piss off mathematicians: solving DLOG in $\Z/n\Z$ is easy but in $(\Z/p\Z)^\times$ is hard
    - But $(\Z/p\Z)^\times$ is isomorphic to $\Z/(p-1)\Z$ so DLOG difficulty must not be preserved over isomorphism
    - Specifically, DLOG is as exactly hard as computing the isomorphism (notice that we send $x \mapsto g^x$)
- DDH is actually easy in $(\Z/p\Z)^\times$, need a subgroup $G \subset (\Z/p\Z)^\times$ with $\abs{G}$ prime
\end{markdown}


\chapter{Quadratic Residues}

\begin{markdown}
## Number Theory Background (09/19)
- Recall: RSA primes are gigantic so it takes time to do operations
    - e.g. picking $e \in (\Z/\phi(n)\Z)^\times$ or finding $d = e^{-1} \pmod{\phi(n)}$ using EEA which runs in a logarithmic number of steps
    - e.g. running $Enc(m) = m^e \pmod n$ or $Dec(c) = c^d \pmod n$ using square-and-multiply which runs in a logarithmic number of steps
- Hard: picking non-squares in integers modulo $p$
    - Set of primes $\abs{((\Z/p\Z)^\times)^2} = \frac{p-1}{2}$ for odd $p > 2$
    - This is because $f(x) = x^2$ is a 2-to-1 function on $(\Z/p\Z)^\times$
        - To prove, show $f(a) = f(b) \iff a = \pm b$
        - Apply Euclid's Lemma: $p \mid (x-y)(x+y)$ implies $p \mid x-y$ or $p \mid x+y$, equivalently, $x = y \pmod p$ or $x = -y \pmod p$
        - Also another theorem: for $R$ integral domain, every polynomial of degree $n$ over $R$ has at most $n$ roots
\end{markdown}

\section{Squares Under a Modulus (09/21)}
The big problem: Given $(\Z/n\Z)^\times$ and $x \in (\Z/n\Z)^\times$, when is $x \equiv \square \pmod n$?

For example, for $\Z/15\Z$, 1 and 4 are squares;
for 8: just 1; for 7: 1, 2, and 4; and for 13: 1, 3, 4, 9, 10, and 12.

This breaks down into cases: $n$ composite, $n$ prime power, $n$ prime

\begin{theorem}
  Suppose $n = \prod p_i^{e_i}$.
  Then, $x \equiv \square \pmod n$ if and only if for all $i$,
  $x \equiv \square \pmod{p_i^{e_i}}$.
\end{theorem}
\begin{prf}
  Suppose $x = y^2 \pmod n$ for a unit $y$.
  Then, $n \mid (x - y^2)$ and $p_i^{e_i} \mid (x-y^2)$ by transitivity.
  That is, $x \equiv y^2 \pmod {p_i^{e_i}}$.
  In the reverse direction, if $p_i^{e_i} \mid (x - y^2)$ for all $i$,
  then by UPF (with some omitted detail), $n \mid (x-y^2)$.
\end{prf}

The prime power case reduces to the prime case
under condtions discovered in the homework problems lol.

\begin{theorem}
  The number of squares in $(\Z/p\Z)^\times$
  is $\frac{p-1}{2}$ for primes $p \geq 3$.
\end{theorem}
\begin{prf}
  This is because $x = y^2 = (-y)^2$ and the size of the set is $p-1$.

  Build a table $(x,g^x)$ instead of $(x,x^2)$:

  For $p=13$ and $g=2$, we get $(1,2,4,8,3,6,12=-1,-2,-4,-8,-3,-6,-12=1)$
  and the squares are the even-indexed values $(1,4,3,12,9,10,1)$.

  This works for tables starting with non-squares:
  in fact, if $g \neq \square$, then $g^3 \neq \square$
  (by the contrapositive, if $g^3 = \square$,
  then $g = \frac{g^3}{g^2} = \frac{\square}{\square} = \square$).

  This gives us the result that $g^x = g^y$ when $x \equiv y \pmod{p-1}$
  (note that this is equivalent to Fermat's Little Theorem,
  the reverse direction requires $g$ coprime to $p-1$).
\end{prf}

\begin{defn}[order]
  $\ord(a)$ is the period of $x \mapsto a^x$ for $a \in (\Z/p\Z)^\times$.

  Equivalently, $\ord(a) = \min\{t \in \Z : a^t = 1, t > 0\}$.
\end{defn}

\begin{lemma}
  Given elements $a$ and $b$, numbers $x$ and $y$:
  \begin{itemize}[nosep]
    \item $a^x = 1$ if and only if $\ord(a) \mid x$
    \item $a^x = a^y$ if and only if $x \equiv y \pmod{\ord(a)}$
    \item $\ord(a^x) = \frac{\ord(a)}{\gcd(x,\ord(a))}$
    \item If $\ord(a)$ and $\ord(b)$ are coprime,
          then $\ord(ab) = \ord(a)\ord(b)$.
  \end{itemize}
\end{lemma}
\begin{prf}
  Only prove the last one:

  Let $t=\ord(a)$, $u=\ord(b)$, $v=\ord(ab)$.
  Then, $(ab)^{tu} = a^{tu}b^{tu} = 1^u 1^t = 1$ so we have $v \mid tu$.
  Now, \Wlog, $(ab)^{vu} = 1^u = 1 \implies a^{vu}b^{vu} = a^{vu}1 = a^{vu} = 1$.
  This gives $t \mid vu$ and $t \mid v$ since $\gcd(t,u)=1$.
  Likewise, $u \mid v$ and we can conclude $tu \mid v$ because $\gcd(t,u)=1$.
  That is, $tu = v$.
\end{prf}


\section{Squares cont'd (09/23)}

\begin{defn}[primitive element]
  $g \in G$ where $\{g^n : n \in \N\} = G$.
  Also called a generator.
\end{defn}

Recall: if there exists primitive $g \in (\Z/p\Z)^\times$,
then for all $h \in (\Z/p\Z)^\times$ where $h=g^k$,
$h \equiv \square \iff \text{$k$ even}$.
We can determine squareness using this fact,
but finding $k$ such that $h = g^k$ is doing a discrete log, which is hard.

Whether or not a primitive element exists is a non-trivial observation:

\begin{theorem}[Gauss' primitive root]
  For all primes $p$, $(\Z/p\Z)^\times$ has a primitive element.
\end{theorem}
\begin{prf}
  Observe that for all polynomials $f(x) \neq 0$ over $\Z/p\Z$,
  the number of roots of $f(x)$ is at most $\deg f$.
  Note that factorization fails in $\Z/n\Z$ in general:
  e.g. $x^2 - 1 = (x-1)(x+1) = (x-3)(x-5)$ mod 8
  or something weird like $x = (3x+2)(2x+3)$ mod 6.
  We have this observation because $\Z/p\Z$ is an integral domain (and indeed, a field).

  Consider $a \in (\Z/p\Z)^\times$.

  Claim $t=\ord(a) \mid p-1$.
  Write $p-1=tq+r$. If $r=0$, done.
  If $r > 0$, $\ord(a) = r < t$, contradiction and indeed $r=0$.

  For each divisor $d$ of $p-1$, consider
  $S_d = \{x \in (\Z/p\Z)^\times : \ord(x) = d\}$.
  Then, $\bigcup_{d \mid p-1}S_d = (\Z/p\Z)^\times$
  and this is a disjoint union.
  To prove Gauss' theorem, we just need $\abs{S_{p-1}} > 0$.

  Proceed in general for arbitrary $\abs{S_d} > 0$ for all $d \mid p-1$.

  If $S_d = \varnothing$, then $\abs{S_d} = 0$.
  Otherwise, claim that $\abs{S_d} = \phi(d) = \abs{(\Z/d\Z)^\times}$.

  If $S_d$ is not empty, then $\exists a \in S_d$ where $\ord(a) = d$.
  Consider $x^d - 1$.
  The roots of this polynomial will include all elements of $S_d$ (and others).
  We can write the set of roots as exactly $\{a^0,\dotsc,a^{d-1}\}$.
  So for all $b \in S_d$, $b = a^k$ since $b$ is a root
  and we need only count those powers with order $d$.
  But that is exactly $\ord(a^i) = \frac{\ord(a)}{\gcd(i,d)} = \frac{d}{\gcd(i,d)}$.
  So we are counting the $i$ such that $\gcd(i,d) = 1$,
  which is exactly $\phi(d)$.

  Now, $p-1 = \abs{(\Z/p\Z)^\times} = \abs{\bigcup_{d \mid p-1}S_d} = \sum\abs{S_d} \leq \sum \phi(d)$ which is equal to $p-1$ by M\"obius inversion.
  That last inequality being an equality implies that
  $\abs{S_d} \neq 0$ for any $d \mid p-1$, and in particular $p-1 \mid p-1$.

  Quick combinatorical proof of this fact:
  write out all the $p-1$ fractions over $p-1$,
  then each of $\phi(d)$ is the number of fractions
  where the denominator reduces to $d$. The sum must be $p-1$.
\end{prf}

\section{Applying to DDH (09/26)}

Recall the Decisional Diffie-Hellman problem:
Given $g$, $g^x$, $g^y$, $g^z$, determine if $z = xy$. Formally, as a game:
\begin{itemize}[nosep]
  \item $\mathcal C$ chooses a bit $b \in \{0,1\}$ and $x,y \xgets{\$}\Z$
  \item $b' \gets \mathcal A(g, g^x, g^y, g^z)$ where $z \gets \begin{cases}xy & b = 0 \\ \xgets{\$}\Z & b = 1\end{cases}$
  \item Win condition: $b = b'$ with non-negligible probability
\end{itemize}
Notice that if $g$ is a primitive root, then $\abs{\{g^x : x \in \Z\}}=p-1$.
But bruteforce DLOG takes $\frac{p-1}{2}$ steps on average.
Then, Elgamal is IND-CPA $\iff$ DDH holds.

\begin{prop}
  The Decisional Diffie-Hellman assumption in $(\Z/p\Z)^\times$
  with a primitive base $g$ does not hold.
\end{prop}
\begin{prf}
  We tell squares and non-squares apart.

  Recall from last lecture's theorem we have that if $g$ is a primitive root,
  $g^x \equiv \square \pmod p \iff x \equiv 0 \pmod 2$.
  Then, by Euler's criterion, $a \equiv \square \pmod p \iff a^{(p-1)/2} \equiv 1 \pmod p$.
  Therefore, it is possible to tell the parity of $x$, $y$, and $z$
  in reasonable time using Euler's criterion (since raising to a power is easy).

  If $xy$ is odd only when $x$ and $y$ are odd,
  so if you know the parity of $z$ you can distinguish if $z=xy$
  or random with non-negligible advantage.
\end{prf}

\begin{prop}[Euler's criterion]
  $a \equiv \square \pmod p \iff a^{(p-1)/2} \equiv 1 \pmod p$
\end{prop}
\begin{prf}
  Suppose $a \equiv \square$ iff $a \equiv g^k$ for even $k=2\ell$
  iff $a^{(p-1)/2} = (g^k)^{(p-1)/2} = g^{k(p-1)/2}=(g^{p-1})^\ell=1^\ell=1$ by \FLT{}.

  Otherwise, $a \not\equiv \square$ iff $a = g^k$ for $k=2\ell+1$
  iff $a^{(p-1)/2} = (g^k)^{(p-1)/2} = g^{(p-1)/2 \cdot (2\ell+1)}=g^{(p-1)/2\cdot 2\ell}\cdot g^{(p-1)/2} = g^{(p-1)/2} \neq 1$.
  But in fact $g^{(p-1)/2} = \sqrt{g^{p-1}} = \sqrt{1} = -1$ since it is not positive 1.
\end{prf}

\begin{corollary}
  For $p > 2$, $-1$ is a square mod $p$ if and only if $p \equiv 1 \pmod 4$.
\end{corollary}
\begin{prf}
  For $-1$ to be a square, we need $(-1)^{(p-1)/2} \equiv 1 \pmod p$.
  That is, $\frac{p-1}{2}$ is even and we have $p \equiv 1 \pmod 4$.
\end{prf}

This quantity $g^{(p-1)/2}$ is useful and we give it a name:

\begin{defn}[Legendre symbol]
  For $p > 2$ and $a \in \Z/p\Z$, the quadratic character of $a$,
  written $(\frac{a}{p}) = a^{(p-1)/2}$,
  is 1 if $a \equiv \square$, 0 if $a \equiv 0$, and -1 if $a \not\equiv \square$.

  Equivalently, define $\chi_p : (\Z/p\Z)^\times \to \{\pm 1\} : a \mapsto (\frac{a}{p})$
  and notice that this is a multiplicative homomorphism that preserves $\chi_p(ab) = \chi_p(a)\chi_p(b)$.
\end{defn}

\begin{theorem}[multiplicativity]
  $(\frac{ab}{p}) = (\frac{a}{p})(\frac{b}{p})$
\end{theorem}
\begin{prf}
  $(\frac{ab}{p}) = (ab)^{(p-1)/2} = a^{(p-1)/2}b^{(p-1)/2} = (\frac{a}{p})(\frac{b}{p})$
\end{prf}

\section{Quadratic Characters in the Complex Plane (09/28)}
Recall: we have that for odd primes, $(\frac{-1}{p})=1 \iff p \equiv 1 \pmod 4$
which we proved by applying Euler's criteron.
We have the similar lemma:

\begin{lemma}
  $(\frac{2}{p}) = 1 \iff p \equiv 1,7 \pmod 8$.
\end{lemma}
\begin{prf}
  This is harder because $2^{(p-1)/2}$ is not easy to analyze, i.e.,
  the order of 2 is not easy to derive.

  What numbers, in general, have finite/known order?
  Complex roots of unity $\zeta_n = e^{2\pi i/n}$.

  We can write $\sqrt{2} = \zeta_8 + \zeta_8^7$,
  so $2^{(p-1)/2} = (\zeta_8 + \zeta_8^7)^{p-1}
    = \frac{(\zeta_8 + \zeta_8^7)^p}{\zeta_8 + \zeta_8^7}$.
  The last transformation is helpful since $p$ powers behave well mod $p$.

  Now, notice that $(x+y)^p \equiv x^p + y^p \pmod p$
  because all the other terms will have a factor of $p \mid \binom{p}{i}$.

  Therefore, $(\frac{2}{p}) \equiv 2^{(p-1)/2} \equiv \frac{\zeta_8^p + \zeta_8^{7p}}{\zeta_8+\zeta_8^7} \pmod p$.

  There are four cases for $p \pmod 8$ because we assume $p > 2$:
  \begin{enumerate}[nosep]
    \item[1.] $\zeta_8^p = \zeta_8^1$ and $\zeta_8^{7p} = \zeta_8^7$
    \item[3.] $\zeta_8^p = \zeta_8^3$ and $\zeta_8^{7p} = \zeta_8^5$
    \item[5.] $\zeta_8^p = \zeta_8^5$ and $\zeta_8^{7p} = \zeta_8^3$
    \item[7.] $\zeta_8^p = \zeta_8^7$ and $\zeta_8^{7p} = \zeta_8^1$
  \end{enumerate}
  Clearly, for $p \equiv 1,7 \pmod 8$, we have
  $\frac{\zeta_8^p + \zeta_8^{7p}}{\zeta_8 + \zeta_8^7}
    = \frac{\zeta_8 + \zeta_8^7}{\zeta_8 + \zeta_8^7} = 1$.
  Slightly less intuitively, for $p \equiv 3,5 \pmod 8$,
  notice that $\zeta_8^3 + \zeta_8^5 = -\sqrt{2}$,
  so the fractions go to $-1$.
\end{prf}

Note: We can algebraically extend $\Z/p\Z$ with the necessary complex numbers
to make the proof valid (or simply assert that the necessary roots of unity exist).

The pattern sort of extends:
\begin{itemize}[nosep]
  \item $(\frac{3}{p}) = 1$ if $p = 1,11 \bmod 12$ and $-1$ if $p = 5,7 \bmod 12$.
  \item $(\frac{5}{p}) = 1$ if $p = \pm 1,\pm 9 \bmod 20$ and $-1$ if $p = \pm 3,\pm 7 \bmod 20$.
  \item $(\frac{7}{p}) = 1$ if $p = \pm 1,\pm 3, \pm 9 \bmod 28$ and $-1$ if $p = \pm 5,\pm 11, \pm 13 \bmod 28$.
\end{itemize}
In fact, we have $(\frac{7}{p}) = 1$ if $p = \pm 1, \pm 9, \pm 25$ mod 28.
This flips the question from is 7 a square mod $p$ to asking if $p$ is a square mod 28.

\begin{lemma}
  $(\frac{-3}{p}) = \begin{cases}1 & p \equiv 1 \pmod 3 \\ -1 & p \equiv 2 \pmod 3\end{cases}$
\end{lemma}
\begin{prf}
  Consider again $(-3)^{(p-1)/2} = (\sqrt{-3})^{p-1}$.
  We can notice $\sqrt{-3} = \sqrt{3} i = \zeta_6 + \zeta_3$.

  This gives us $(\sqrt{-3})^{p-1} = \frac{\zeta_3^p - \zeta_3^{2p}}{\zeta_3 - \zeta_3^2}$
  because $\zeta_6 = -\zeta_3^2$.

  If $p \equiv 1 \pmod 3$, then $\frac{\zeta_3^p - \zeta_3^{2p}}{\zeta_3 - \zeta_3^2}
    = \frac{\zeta_3 - \zeta_3^{2}}{\zeta_3 - \zeta_3^2} = 1$
  and if $p \equiv 2 \pmod 3$, $\frac{\zeta_3^p - \zeta_3^{2p}}{\zeta_3 - \zeta_3^2}
    = \frac{\zeta_3^2 - \zeta_3^1}{\zeta_3 - \zeta_3^2} = -1$.
\end{prf}

Notice that to get to $\sqrt{3}$ on the complex plane, we need $\zeta_{12}$,
which explains why we see mod 12 in the rule.
To get $\sqrt{5}$, we can either use the fact that $\cos \frac{2\pi}{5} = \frac14(\sqrt{5}-1)$
or notice that $(\zeta_5 - \zeta_5^2 - \zeta_5^3 + \zeta_5^4)^2
  = (4 - \zeta_5 - \zeta_5^2 - \zeta_5^3 - \zeta_5^4)
  = 5 - (1 + \zeta_5^1 + \zeta_5^2 + \zeta_5^3 + \zeta_5^4) = 5$.
We can then execute the same fraction-by-cases technique, getting our result mod 5.

Aside: This is the Gauss sum for $\sqrt{5} = \sum (\frac{i}{5})\zeta_5^i$.

\section{Quadratic Reciprocity (09/30)}

Recall the pattern from last lecture,
where we noticed that asking if $q$ is a square mod $p$
seems to be like asking if $p$ is a square mod $4q$.
This is almost true, but in fact

\begin{theorem}[Quadratic Reciprocity]
  $(\frac{q}{p}) = (\frac{p}{q})$ for odd primes $p \neq \pm q$
  where at least one is congruent to 1 mod 4 and at least one is positive.
\end{theorem}

Equivalently, for all distinct positive odd primes $p$ and $q$,
$(\frac{p}{q})(\frac{q}{p}) = (-1)^{\frac{p-1}{2}\frac{q-1}{2}}$

The proof follows by Gauss sums and the vague ideas from the last lecture.

This means we can evaluate any Legendre symbol using a modulus as either one of
\begin{align*}
  \qty(\frac{-1}{p}) = (-1)^{\frac{p-1}{2}}
   & = \begin{cases}1 & p \equiv 1 \pmod 4 \\ -1 & p \equiv 3 \pmod 4\end{cases}         \\
  \qty(\frac{2}{p})  = (-1)^{\frac{p^2-1}{8}}
   & = \begin{cases}1 & p \equiv \pm 1 \pmod 8 \\ -1 & p \equiv \pm 3 \pmod 8\end{cases} \\
  \qty(\frac{q}{p})  = \qty(\frac{p}{q})(-1)^{\frac{p-1}{2}\frac{q-1}{2}}
   & = \begin{cases*}
         (\frac{p}{q})  & $p \equiv \pm 1 \pmod 4$ or $q \equiv \pm 1 \pmod 4$ \\
         -(\frac{p}{q}) & $p \equiv q \equiv 3 \pmod 4$
       \end{cases*}
\end{align*}
which is nicer than using Euler's criterion.

\begin{example}
  Is 71 a square mod 101?
\end{example}
\begin{sol}
  Write $\leg{71}{101} = \leg{101}{71} = \leg{30}{71} = \leg{2}{71}\leg{3}{71}\leg{5}{71}$
  by quadratic reciprocity and multiplicativity.

  Then, $\leg{2}{71} = 1$ since $71 \equiv 7 \pmod 8$.

  Also, $\leg{3}{71} = -\leg{71}{3} = -\leg{2}{3} = 1$ since $71 \equiv 3 \pmod 4$.

  Finally, $\leg{5}{71} = \leg{71}{5} = \leg{1}{5} = 1$ since 1 is always a square.

  This gives $\leg{71}{101} = 1\cdot 1 \cdot 1 = 1$ so 71 is a square mod 101.
\end{sol}

Asymptotically, this is not faster than Euler's criterion
because we require factoring. However, it is prettier.

To deal with a random large number, we must consider what to do
after factoring out all the 2s (since we can deal with those quickly).

\begin{defn}[Jacobi symbol]
  For all $m,n \in \N_{>0}$ with $n$ odd, $\leg{m}{n} = \prod_{i=1}^k \leg{m}{p_i}$
  where $\prod_{i=1}^k p_i = n$ is the prime factorization of $n$
\end{defn}

\begin{theorem}[Jacobi]
  For all positive and odd $m$ and $n$,
  \begin{align*}
    \qty(\frac{-1}{n}) = (-1)^{\frac{n-1}{2}}
     & = \begin{cases}1 & n \equiv 1 \pmod 4 \\ -1 & n \equiv 3 \pmod 4\end{cases}         \\
    \qty(\frac{2}{n})  = (-1)^{\frac{n^2-1}{8}}
     & = \begin{cases}1 & n \equiv \pm 1 \pmod 8 \\ -1 & n \equiv \pm 3 \pmod 8\end{cases} \\
    \qty(\frac{m}{n})  = \qty(\frac{n}{m})(-1)^{\frac{n-1}{2}\frac{m-1}{2}}
     & = \begin{cases}
           (\frac{n}{m})  & n \equiv \pm 1 \pmod 4 \text{ or } m \equiv \pm 1 \pmod 4 \\
           0              & \gcd(m,n) \neq 1                                          \\
           -(\frac{n}{m}) & n \equiv m \equiv 3 \pmod 4
         \end{cases}
  \end{align*}
\end{theorem}

Note: For Legendre symbols, $\leg{a}{p} = 1 \iff a \equiv \square \pmod p$.
However, for Jacobi symbols, we only have the one-way implication
$\leg{m}{n} = -1 \implies m \not\equiv \square \pmod n$.

Return now to the application to cryptography,
specifically to Goldwasser--Micali.

\subsubsection{Goldwasser--Micali cryptosystem}

\textbf{Key Generation: } Choose random primes $p$, $q$. Set $n = pq$.

Choose $x \in (\Z/n\Z)^\times$ such that
$\leg{x}{p} = \leg{x}{q} = -1$ (then $\leg{x}{n} = 1$).
Publish $x$.

\textbf{Encrypt:} $m \in \{0,1\}$

Choose some $r \xgets{\$} (\Z/n\Z)^\times$.
Then, $Enc(m) = x^m r^2 = c$.

\textbf{Decrypt:} Determine whether $c$ is a ``fake'' square using the factorization.

The underlying assumption is that it is not easy to distinguish
actual squares mod $n$ and ``fake'' squares mod $n$.

\chapter{Primality}

\section{Primality Testing (10/03)}

Given $n \in \Z$, how can we tell if $n$ is prime?

\begin{lemma}[Fermat test]
  Recall \FLT{}: for a prime $p$, $a \in (\Z/p\Z)^\times \implies a^{p-1} = 1$.
  Therefore, if $a \in (\Z/n\Z)^\times$ and $a^{n-1} \neq 1$,
  then $n$ is not prime.
\end{lemma}

\begin{defn}[Fermat witness]
  Let $n \in \N$, $\alpha \in (\Z/n\Z)^\times$ where $\alpha^{n-1} \neq 1$.
\end{defn}

When $n$ is prime, no Fermat witness can exist.
When $n$ is not prime, only some elements are Fermat witnesses.
The other elements are \emph{Fermat liars}.
How many liars are in $(\Z/n\Z)^\times$?

\begin{theorem}
  For $n > 2$, if there exists one Fermat witness in $(\Z/n\Z)^\times$,
  then there exist at least $\frac{\phi(n)}{2}$ Fermat witnesses.
\end{theorem}
\begin{prf}
  Consider the set $H=\{ \alpha \in \mgrp{n} : \alpha^{n-1} = 1\}$.

  $H$ is a subgroup: $1 \in H$, $ab \in H$, $a^{-1} \in H$ (trivial by exponentiation properties).

  So by Lagrange's theorem, $\abs{H} \mid \abs{\mgrp{n}}$.

  Either (1) $\abs{H} = \phi(n)$, so there are no witnesses,
  or (2) $\abs{H} < \phi(n)$, so $\abs{H} \leq \frac{\phi(n)}{2}$.
\end{prf}

\begin{defn}[Carmichael number]
  $n \in \N$, $n > 2$ such that $n$ is composite and $n$ has no Fermat witnesses.
\end{defn}
Examples: $n = 561 = 3 \times 11 \times 17$.
By \FLT{}, we have $\alpha^{n-1} = \alpha^{560}$ is 1 mod 3, 1 mod 11, and 1 mod 17.

Recall that for $n$ prime:
$a^{\frac{n-1}{2}} \equiv \leg{a}{n} \pmod n$ when $n > 2$, odd, and $a \in \mgrp{n}$.
This gives us the following test:

\begin{lemma}[Solovay--Strassen test]
  If $a^{\frac{n-1}{2}} \not\equiv \leg{a}{n} \pmod n$, then $n$ is not prime.
\end{lemma}

We can calculate $a^{\frac{n-1}{2}}$ by repeated squaring
and $\leg{a}{n}$ by Jacobi reciprocity and factoring out 2's.
We can now define witneses as in the Fermat test.

\begin{defn}[Euler (Solovay--Strassen) witness]
  An element $\alpha \in \mgrp{n}$ where
  $\leg{\alpha}{n} \not\equiv \alpha^{\frac{n-1}{2}} \pmod n$.
  If an element is not an Euler witness, it is an Euler liar.
\end{defn}

Notice that all Euler witnesses must also be Fermat witnesses,
meaning that hopefully we have a more refined test here.

\begin{theorem}
  If $n > 2$ is composite and odd, then there exists at least one Euler witness.
\end{theorem}
\begin{prf}
  Suppose $n$ is composite and $n = p \times k$.

  If $p \nmid k$, then solve $\alpha \equiv \beta \pmod p$ and $\alpha \equiv 1 \pmod k$
  where $\beta$ is a quadratic non-residue mod $p$.
  Now, calculate
  \[
    \leg{\alpha}{n} = \leg{\alpha}{p}\leg{\alpha}{k} = \leg{\beta}{p}\leg{1}{k} = (-1)(1) = -1
  \]
  Suppose $\alpha^{\frac{n-1}{2}}$ is $-1$.
  Then, $\alpha^{\frac{n-1}{2}} \equiv -1 \pmod n$
  and that means $\alpha^{\frac{n-1}{2}} \equiv -1 \pmod k$.
  But we know $\alpha \equiv 1 \pmod k$, so this is a contradiction.

  Otherwise, $p \mid k$.
  Let $\alpha = 1 + k$. Calculate
  \[
    \leg{\alpha}{n} = \leg{1+k}{n} = \leg{1+k}{p}\leg{1+k}{k} = \leg{1}{p}\leg{1}{k} = (1)(1) = 1
  \]
  Suppose $\alpha^{\frac{n-1}{2}} = 1$.
  This implies that $\ord(\alpha) \mid \frac{n-1}{2}$.
  Calculate $\alpha^p = (1+k)^p = 1^p + \underbrace{p k^1 + \dotsb + \binom{p}{p} k^p}_{0 \pmod n} = 1$
  which implies $\ord(\alpha) = p$.
  But $p \mid n \implies p \nmid n-1 \implies p \nmid \frac{n-1}{2}$.

  Therefore, $\alpha$ is an Euler witness.
\end{prf}

This theorem combined with the at-least-$\frac{\phi(n)}{2}$ theorem
means that we have for every odd, composite $n > 2$ there are $\frac{\phi(n)}{2}$ Euler witnesses.

\section{Strong Primality Testing (10/05)}

Recall: for the Fermat test, evaluate $a^{n-1}$ a bunch of times.
If it is equal to 1, prime or liar; otherwise, composite.
For the Solovay--Strassen test, evaluate $a^{\frac{n-1}{2}} = \leg{a}{n}$.
If yes, prime or Euler liar; otherwise, composite.
Also, there are an infinite number of Carmichael numbers that screw with this
but otherwise you have around a 50\% chance of getting a witness.

We can refine this further beyond considering $n-1$ and $\frac{n-1}{2}$.

Write $n-1 = 2^t \cdot s$ so that $s$ is odd.
Then, $a^{n-1}$ is $a^s$ squared $t$ times.
So instead of asking if $a^{2^t s} = 1$, consider if $a^{2^{t-1} s}$
is an ``expected'' square root of 1, i.e., $\pm 1$.
If it is not, it is composite.
If it is and it is $-1$, we have a prime or liar.
If it is and it is 1, keep going back.
If we reach $a^s=1$, we get no information.

\begin{lemma}[Miller--Rabin test]
  Let $x \gets a^s$. Do:
  \begin{itemize}[nosep]
    \item If $x = 1$, stop. Probably prime.
    \item If $x = -1$, stop. Probably prime.
    \item Otherwise, $x \gets x^2$
  \end{itemize}
  while $x \neq a^{2^t s}$. If we reach the end, it is composite.
\end{lemma}

\begin{defn}[Miller--Rabin (strong) liar]
  $a \in \mgrp{n}$ if either $a^s = 1$ or $a^{2^k s} = -1$ for $0 \leq k < t$.
\end{defn}

We call this a ``strong liar'' because every strong liar is an Euler liar,
and every Euler liar is a Fermat liar.

\begin{theorem}
  Suppose $n$ has at least two distinct prime factors.
  Then, the number of Miller--Rabin liars is at most $\frac{\phi(n)}{4}$
  and in general, if $n$ has $\ell$ distinct prime factors,
  there are at most $\frac{\phi(n)}{2^\ell}$ Miller--Rabin liars.
\end{theorem}

We can make these primality tests deterministic by iterating $a = 1,\dotsc,n$.
We do not need to go to $a=n$ and instead
we can establish an upper bound on the smallest witness.
The bound (by Bach) is $O(\log^2 n)$, specifically, $2\log^2n$.
But this requires the Generalized Riemann Hypothesis
which everyone believes anyways, so we just check $a = 1,\dotsc,2\log^2 n$.

To analyze complexity, notice that we have $\log n$ multiplications at each step,
i.e., $\log^{1+\epsilon} n$ bit operations using fast multiplication.
So the complexity is $O(\log^{2 + (1 + \epsilon) + 1} n)$.

Further reading:
\begin{itemize}[nosep]
  \item AKS (Agrawal--Kayal--Saxena; 2004) primality test in $O(\log^6 n)$
        which does not rely on GRH and was an undegrad project(!!)
  \item ECPP (elliptic curve prime proving) notable for not having liars,
        also does not require GRH and runs non-deterministically (Monte Carlo)
        in $O(\log^5 n)$
  \item Cyclotomic primality test in $O((\log n)^{\log \log n})$,
        best until AKS proved that primality is in P.
\end{itemize}

Since there are $\frac{n}{\log n} + O(\sqrt n)$ primes less than $n$,
we can pick random numbers of size $e^\ell$ to get an approximate $\frac{1}{\ell}$
probability of a prime.

\end{document}