\RequirePackage{chngcntr}
\newcommand{\setsectiontitle}[1]{
  \titleformat{\section}{\normalfont\large\bfseries}{#1~\thesection}{1em}{}
  \settowidth{\cftsecnumwidth}{#1~0000~}
  \renewcommand\cftsecpresnum{#1~}
  \addtocontents{toc}{\protect\renewcommand*{\protect\cftsecpresnum}{#1~}}
}
\bookmarksetup{numbered}

\setlength\topsep{6pt}

%
% Theorem boxes (taken from vEnhance/napkin)
%

\usepackage{mdframed}
\usepackage{amsthm,thmtools}
\declaretheoremstyle[
  spaceabove=6pt,spacebelow=6pt,
  headfont=\normalfont\itshape,bodyfont=\normalfont,
  qed=\qedsymbol]{proof}
\declaretheorem[name=Proof,style=proof,unnumbered]{prf}
\declaretheorem[name=Solution,style=proof,unnumbered]{sol}

\mdfdefinestyle{mdround}{
	roundcorner=10pt,
	linewidth=1pt,
	skipabove=12pt,
  innertopmargin=12pt,
	innerbottommargin=9pt,
	skipbelow=2pt,
	nobreak=true,
}
\mdfdefinestyle{mdsquare}{
	linewidth=0.5pt,
	skipabove=12pt,
  innertopmargin=12pt,
	innerbottommargin=9pt,
	skipbelow=2pt,
	nobreak=true,
}
\mdfdefinestyle{mdquote}{
	skipabove=8pt,
  innertopmargin=12pt,
	innerbottommargin=9pt,
	linewidth=2pt,
	rightline=false,
	leftline=true,
	topline=false,
	bottomline=false,
}

% Results (purple)
\declaretheoremstyle[
	headfont=\bfseries\color{Purple},
	mdframed={style=mdround,linecolor=Mulberry,backgroundcolor=Orchid!5},
	headpunct={\\[3pt]},
	postheadspace={0pt}
]{thmroundresult}
\declaretheoremstyle[
	headfont=\bfseries\color{Violet},
	mdframed={style=mdsquare,linecolor=Violet,backgroundcolor=Periwinkle!5},
	headpunct={\\[3pt]},
	postheadspace={0pt}
]{thmsquareresult}
\declaretheoremstyle[
	headfont=\bfseries\color{MidnightBlue},
	mdframed={style=mdsquare,linecolor=RoyalBlue,backgroundcolor=CornflowerBlue!5},
	headpunct={\\[3pt]},
	postheadspace={0pt}
]{thmprop}
\declaretheoremstyle[
	headfont=\bfseries\color{MidnightBlue},
  notefont=\mdseries,
	bodyfont=\normalfont,
	mdframed={style=mdquote,linecolor=RoyalBlue,backgroundcolor=CornflowerBlue!5},
]{thmquoteresult}
\declaretheorem[refname={Theorem,Theorems},style=thmroundresult,numberwithin=section]{theorem}
\declaretheorem[refname={Lemma,Lemmas},style=thmsquareresult,sibling=theorem]{lemma}
\declaretheorem[name=Proposition,refname={Proposition,Propositions},style=thmprop,sibling=theorem]{prop}
\declaretheorem[refname={Corollary,Corollaries},style=thmquoteresult,sibling=theorem]{corollary}

% Problems (red)
\declaretheoremstyle[
	headfont=\bfseries\color{RawSienna},
	mdframed={style=mdround,linecolor=Red,backgroundcolor=CarnationPink!2},
	headpunct={\\[3pt]},
	postheadspace={0pt}
]{thmroundred}
\declaretheoremstyle[
	headfont=\bfseries\color{RedViolet},
	mdframed={style=mdround,linecolor=Magenta,backgroundcolor=CarnationPink!2},
	headpunct={\\[3pt]},
	postheadspace={0pt}
]{thmroundpink}
\declaretheoremstyle[
	headfont=\bfseries\color{RawSienna},
  notefont=\mdseries,
	bodyfont=\normalfont,
	mdframed={style=mdquote,linecolor=RawSienna,backgroundcolor=CarnationPink!2},
]{thmquotered}
\declaretheorem[refname={Conjecture,Conjectures},style=thmroundpink,sibling=theorem]{conjecture}
\declaretheorem[refname={Problem,Problems},style=thmroundred,sibling=theorem]{problem}
\declaretheorem[name=Exercise,refname={Exercise,Exercises},style=thmquotered,sibling=theorem]{xca}

% Definitions (green)
\declaretheoremstyle[
	headfont=\bfseries\color{OliveGreen},
	bodyfont=\normalfont,
	headpunct={\\[3pt]},
	postheadspace={0pt},
	mdframed={style=mdsquare,linecolor=ForestGreen,backgroundcolor=ForestGreen!5},
]{thmsquaregreen}
\declaretheorem[refname={definition,definitions},style=thmsquaregreen,sibling=theorem]{definition}
\NewDocumentEnvironment{defn}{oO{#1}}{\IfNoValueTF{#1}{\definition}{\definition[#1]\index{#2}}}{\enddefinition}
\NewDocumentEnvironment{defn*}{o}{\IfNoValueTF{#1}{\definition}{\definition[#1]}}{\enddefinition}

% Misc (black)
\declaretheoremstyle[
	headfont=\bfseries,
  notefont=\mdseries,
	bodyfont=\normalfont,
	mdframed={style=mdquote,linecolor=black,backgroundcolor=RedViolet!5!gray!5}
]{thmquoteblack}
\declaretheoremstyle[
	headfont=\bfseries\color{Brown},
  notefont=\mdseries,
	bodyfont=\normalfont,
	mdframed={style=mdquote,linecolor=Goldenrod,backgroundcolor=Goldenrod!5}
]{thmquoteyellow}
\declaretheorem[refname={Remark,Remarks},style=thmquoteyellow,sibling=theorem]{remark}
\declaretheorem[refname={Claim,Claims},style=thmquoteblack,sibling=theorem]{claim}
\declaretheorem[refname={Example,Examples},style=thmquoteblack,sibling=theorem]{example}

% \skipto{5} immediately before environment #5
\newcommand{\skipto}[2][theorem]{\setcounter{#1}{#2}\addtocounter{#1}{-1}}

% \RequirePackage[hybrid,underscores=false]{markdown}

% Use inside of definitions: "a \term{thing} is a whatever"
\makeindex[columns=3, title=Index of Defined Terms, intoc]
\indexsetup{level=\chapter*,toclevel=section}

% For thmbox (see tex.se/41100)
\RequirePackage{tablefootnote}
\newcommand{\spewnotes}{\tfn@tablefootnoteprintout\global\let\tfn@tablefootnoteprintout\relax\gdef\tfn@fnt{0}}

% List of lectures (marked in margin)
\RequirePackage{multicol,xpatch}
\RequirePackage{marginnote}
\newlistof{lecture}{lec.toc}{Lectures}
\xpatchcmd{\listoflecture}{\chapter*}{\section*}{}{}
\settowidth{\cftlecturenumwidth}{Lecture~0000~}
\newcommand{\lecture}[1]{%
  \refstepcounter{lecture}%
  \addcontentsline{lec.toc}{lecture}{\protect\numberline{Lecture~\thelecture}#1}%
  \marginnote{\normalsize{\textit{Lecture \thelecture\\#1}}}}
