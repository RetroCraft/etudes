\documentclass[notes,tikz]{agony}
\usetikzlibrary{cd}
\renewcommand{\C}{\cat{C}}
\newcommand{\D}{\cat{D}}
\newcommand{\E}{\cat{E}}
\newcommand{\comma}{\mathbin{\downarrow}}
\DeclareMathOperator{\dom}{dom}
\DeclareMathOperator{\cod}{cod}

\title{\emph{Category Theory in Context} Exercises}
\renewcommand{\thexca}{\thesection.\roman{theorem}}
\newcommand{\inenum}{\vspace{-1.8em}}
\setlist[enumerate,1]{(i)}

\begin{document}
\thispagestyle{firstpage}
\renewcommand{\contentsname}{Exercises\\{\Large from Riehl, \emph{Category Theory in Context}}}
\tableofcontents

I am self-studying this alongside Aluffi, \emph{Algebra:\ Chapter 0},
so there is a bit of mixed notation.

\chapter{Categories, Functors, Natural Transformations}

\section{Abstract and concrete categories}

\begin{xca}
  \leavevmode
  \begin{enumerate}
    \item Consider a morphism $f : x \to y$.
          Show that if there exists a pair of morphisms $g, h : y \toto x$
          so that $g f = 1_x$ and $f h = 1_y$, then $g = h$ and $f$ is an isomorphism.
    \item Show that a morphism can have at most one inverse isomorphism.
  \end{enumerate}
\end{xca}
\begin{enumerate}
  \item \begin{prf}\inenum
          We can compose together $gfh = (gf)h = 1_Xh = h$.
          But $gfh = g(fh) = g1_y = g$ by associativity.
          Therefore, $g = h$.
          Since $gf = 1_x$ and $fh = fg = 1_y$, $f$ is an isomorphism.
        \end{prf}
  \item \begin{prf}\inenum
          Suppose $f : x \to y$ has two inverses $g$ and $h$.
          Then, as above, $g = g1_y = g(fh) = (gf)h = 1_xf = h$ by associativity.
        \end{prf}
\end{enumerate}

\begin{xca}
  Let $\C$ be a category.
  Show that the collection of isomorphisms in $\C$ defines a subcategory,
  the maximal groupoid inside $\C$.
\end{xca}
\begin{prf}
  Let $X$, $Y$, and $Z$ be objects in $\C$.
  We must show identity and composition, since we get associativity for free from $\C$.

  The identity morphisms $1_X$ are isomorphisms, since they are their own inverses.
  Therefore, they are present in the maximal groupoid.

  Now, if $f : X \to Y$ and $g : Y \to Z$ are isomorphisms,
  then the composition $gf : X \to Z$ is an isomorphism with inverse $f^{-1}g^{-1}$
  since $gff^{-1}g^{-1} = gg^{-1} = 1_Z$ and $f^{-1}g^{-1}gf = f^{-1}f = 1_X$.

  Therefore, the maximal groupoid is in fact a category.
\end{prf}

\begin{xca}\label{xca:slice}
  For any category $\C$ and any object $c \in \C$, show that:
  \begin{enumerate}
    \item There is a category $c/\C$ whose objects are morphisms
          $f : c \to x$ with domain $c$ and
          in which a morphism from $f : c \to x$ to $g : c \to y$
          is a map $h : x \to y$ between the codomains so that the triangle
          \begin{center}
            \begin{tikzcd}[column sep=small]
                            & c \dlar["f"] \drar["g"] &   \\
              x \ar[rr,"h"] &                         & y
            \end{tikzcd}
          \end{center}
          commutes, i.e., so that $g = hf$.
    \item There is a category $\C/c$ whose objects are morphisms
          $f : x \to c$ with codomain $c$ and
          in which a morphism from $f : x \to c$ to $g : y \to c$
          is a map $h : x \to y$ between the codomains so that the triangle
          \begin{center}
            \begin{tikzcd}[column sep=small]
              x \ar[rr,"h"]\drar["f"] &   & y \dlar["g"] \\
                                      & c &
            \end{tikzcd}
          \end{center}
          commutes, i.e., so that $f = gh$.
  \end{enumerate}
  The categories $c/\C$ and $\C/c$ are called slice categories of $\C$ under and over $c$, respectively.
\end{xca}
\begin{enumerate}
  \item See Exercise I.3.7, Aluffi.
  \item See Example I.3.5, Aluffi.
\end{enumerate}

\section{Duality}

\begin{xca}
  Defining $\C/c$ to be $(c/(\C\op))\op$,
  deduce Exercise 1.1.iii(ii) from Exercise 1.1.iii(i).
\end{xca}
\begin{prf}
  We must establish that $\C/c$ is in fact $(c/(\C\op))\op$.
  Then, everything (inverses, composition, associativity) follows immediately from duality.

  First, notice that in $c/(\C\op)$,
  we have objects that are morphisms in $\C\op$, i.e.,
  the same that are in $\C$ but backwards:
  \begin{center}
    \begin{tikzcd}[column sep=small]
                       & c \dlar["f\op" above] \drar["g\op"] &   \\
      x \ar[rr,"h\op"] &                                     & y
    \end{tikzcd}
  \end{center}
  To get to the desired commutative diagram, we have to apply the opposite operation once more.
\end{prf}

\begin{xca}
  \leavevmode
  \begin{enumerate}
    \item Show that a morphism $f : x \to y$ is a split epimorphism
          in a category $\C$ if and only if for all $c \in \C$,
          the post-composition function $f_* : \C(c,x) \to \C(c,y)$ is surjective.
    \item Argue by duality that $f$ is a split monomorphism if and only if
          for all $c \in \C$, the pre-composition function
          $f^* : \C(y,c) \to \C(x,c)$ is surjective.
  \end{enumerate}
\end{xca}
\begin{enumerate}
  \item \begin{prf}\inenum
          ($\Rarr$) Suppose $f$ is a split epimorphism, i.e.,
          it is a left inverse of some morphism $g : y \to x$.
          That is, $fg = 1_y$.

          Let $h : c \to y$.
          We must show $f_*(j) = fj = h$ for some $j : c \to x$.
          Notice that $f(gh) = (fg)h = 1_y h = h$.
          Therefore, if we let $j := gh$, we are done.

          ($\Larr$) Suppose $f_*$ is surjective, i.e.,
          for all $g : c \to y$ there exists an $h : c \to x$
          such that $f_*(h) = fh = g$.
          In particular, for $g' = 1_y$, there exists $h' : y \to x$
          such that $fh' = 1_y$.
          That is, $f$ is a split epimorphism, the retraction of $h'$.
        \end{prf}
  \item \begin{prf}\inenum
          Apply part (i) to the category $\C\op$:
          \begin{quote}
            $f\op \in \C\op(x,y)$ is a split epimorphism
            if and only if for all $c \in \C\op$,
            the post-composition function $f_*\op : \C\op(c,x) \to \C\op(c,y)$ is surjective.
          \end{quote}
          But this is exactly the same as saying
          \begin{quote}
            $f \in \C(y,x)$ is a split monomorphism
            if and only if for all $c \in \C$,
            the pre-composition function $f^* : \C\op(x,c) \to \C\op(y,c)$ is surjective.
          \end{quote}
          because $\C\op(x,y) = \C(y,x)$
          and a split epimorphism $f\op g\op = 1_y$
          becomes a split monomorphism $gf = 1_y$.
        \end{prf}
\end{enumerate}

\begin{xca}
  Prove Lemma 1.2.11 by proving either (i) and (ii), then arguing by duality.
  Conclude that the monomorphisms in any category define a subcategory of that category
  and dually that the epimorphisms also define a subcategory.
\end{xca}
\begin{prf}
  (i) Let $f : x \mono y$ and $g : y \mono z$.
  We must show that $gf$ is monic, i.e.,
  $gfh = gfk \implies h = k$ for all $h,k : c \toto x$.
  Suppose $gfh = gfk$.
  Since $g$ is monic, we know that $g(fh) = g(fk) \implies fh = fk$.
  But then, since $f$ is monic, $fh = fk \implies h = k$.

  (ii) Let $f : x \to y$ and $g : y \to z$ such that $gf : x \mono z$.
  We must show $f$ is monic, i.e., $fh = fk \implies h = k$ for all $h,k : c \toto x$.
  Suppose $fh = fk$.
  Then, by pre-composing $g$, $(gf)h = (gf)k$.
  Since $gf$ is monic, $h = k$.

  Now, we can show that the subcategory $\cat{C_{Monic}}$ of the same objects
  as $\C$ and with only its monomorphisms is a category.
  First, since for all $h,k \in \C(c,x)$,
  $1_xh = 1_xk \implies h = k$, we have the identities in $\cat{C_{Monic}}$.
  Then, due to (i), the compositions of all morphisms are in the subcategory.
  Since the objects are unchanged, we can conclude $\cat{C_{Monic}}$ is a subcategory.

  Dually, the opposite monomorphism $f\op$ in $\C\op$ such that
  $f\op h\op = f\op k\op \implies h\op = k\op$ is the epimorphism
  $f$ in $\C$ such that $hf = kf \implies h = k$.
  Therefore, the dual of (i) is (i') and the dual of (ii) is (ii').
  Finally, the dual of the existence of the subcategory $\cat{C_{Monic}}$
  is a subcategory $\cat{C_{Epic}}$ of epimorphisms.
\end{prf}

\skipto{6} % TODO: requires ring theory
\begin{xca}
  Prove that a morphism that is both a monomorphism
  and a split epimorphism is necessarily an isomorphism.
  Argue by duality that a split monomorphism that is
  an epimorphism is also an isomorphism.
\end{xca}
\begin{prf}
  Suppose $f : x \mono y$ is also a split epimorphism such that $fg = 1_y$ for some $g : y \to x$.
  Then, since $f$ is monic, $f(gf) = (fg)f = (1_y)f = f = f(1_x) \implies gf = 1_x$.
  That is, $f$ is an isomorphism.

  By duality, a split epimorphism $f\op g\op = 1_y$ is a split homomorphism $gf = 1_y$.
  Therefore, a split homomorphism that is an epimorphism is an isomorphism too.
\end{prf}

\begin{xca}
  Regarding a poset $(\cat P, \leq)$ as a category,
  define the supremum of a subcollection of objects $a \in \cat P$
  in such a way that the dual statement defines the infimum.
  Prove that the supremum of a subset of objects is unique, whenever it exists,
  in such a way that the dual proof demonstrates the uniqueness of the infimum.
\end{xca}
\begin{prf}
  Let $a$ be the object such that $f_x : x \to a$ exists for all other $x$ in the subcollection.
  That is, for every object $x$, $y \leq a$.
  Then, the dual definition is the object $a\op$ in $\cat P\op$
  such that $f\op_X : x \to a$ always exists, i.e.,
  the object $a'$ in $\cat P$ such that $f_x : a \to x$ always exists.

  Suppose there are two suprema $a$ and $a'$.
  Then, by the definition for $a$, we just have a morphism $a \to a'$.
  But if $a'$ is a supremum, there must also be a morphism $a' \to a$.
  Therefore, $a \leq a' \leq a$, which means $a = a'$ since $\cat P$ is a poset
  (and not a pre-ordered set).
\end{prf}

\section{Functoriality}

\begin{xca}
  What is a functor between groups, regarded as one-object categories?
\end{xca}
\begin{sol}
  A functor $F : \cat G \to \cat H$ maps the single object
  $FG = H$ and each morphism (i.e., group element)
  such that for all $a,b : G \toto G$, $F(ab) = (Fa)(Fb)$ and $Fe_G = e_H$.

  That is, in the language of groups, a functor is a group homomorphism.
\end{sol}

\begin{xca}
  What is a functor between preorders, regarded as categories?
\end{xca}
\begin{sol}
  A functor $F : (\cat P, \leq) \to (\cat Q, \preccurlyeq)$
  sends the objects of $\cat P$ to objects in $\cat Q$
  such that if $a \leq b$, then $Fa \preccurlyeq Fb$.
  That is, we can regard $F$ as a set function $P \to Q$
  that is increasing with respect to the respective preorders.
\end{sol}

\begin{xca}
  Find an example to show that the objects and morphisms in the image of a functor
  $F : \C \to \D$ do not necessarily define a subcategory of $\D$.
\end{xca}
\begin{sol}
  Let $\C$ be a groupoid with two groups
  generated by two elements $A = \ev{a}$ and $B = \ev{b}$:
  \begin{center}
    \begin{tikzcd}
      A \arrow["a"', loop, distance=2em, in=305, out=235] & B \arrow["b"', loop, distance=2em, in=305, out=235]
    \end{tikzcd}
  \end{center}
  Let $\D$ be a groupoid with one group $G = \ev{\alpha,\beta}$.
  Now, let $F : \C \to \D$ with $FA = FB = G$,
  $Fa = \alpha$, and $Fb = \beta$.

  The image of $F$ is not a category.
  Both $\alpha$ and $\beta$ are in the image, but their composition $\beta\alpha$ is not
  since there was no $ba$ composition in $\C$.
\end{sol}

\begin{xca}
  Verify that the constructions introduced in Definition 1.3.11
  (functors $\C(c,-)$ and $\C(-,c)$ represented by $c$) are functorial.
\end{xca}
\begin{prf}
  Recall that $\C(c,-) : \C \to \cat{Set}$
  sends $x$ to the set $\C(c,x)$
  and $f : x \to y$ to the post-composition function
  $f_* : \C(c,x) \to \C(c,y) : h \mapsto fh$.
  Verify the functoriality axioms:

  Let $f : x \to y$ and $g : y \to z$ be composable morphisms where $gf : x \to z$.
  Then, $(gf)_* : \C(c,x) \to \C(c,z)$.
  We can verify that $(gf)_*(h) = (gf)h = g(f(h)) = g(f_*h) = g_*f_*h$ for all $h : c \to z$.
  That is, $(gf)_* = g_*f_*$.

  Consider an identity $1_x : x \to x$.
  Then, the post-composition function $(1_x)_* : \C(c,x) \to \C(c,x)$
  sends morphisms to themselves because $h1_x = h$ and $1_xg = g$ by definition,
  i.e., it is $1_{\C(c,x)}$.
\end{prf}

\begin{xca}
  What is the difference between a functor $\C\op \to \D$
  and a functor $\C \to \D\op$?
  What is the difference between a functor $\C \to \D$
  and a functor $\C\op \to \D\op$?
\end{xca}
\begin{sol}
  If a contravariant functor $F : \C\op \to \D$
  sends $a \mapsto x$ and $f\op : b \to a$ to $g : x \to y$,
  then the ``same'' functor $F' : \C \to \D\op$
  sends $a \mapsto x$ and $f : a \to b$ to $g\op : y \to x$.
  For the functoriality axioms to hold,
  we must have that $F' f = F f\op$ for all morphisms.
  That is, we can consider $F'$ as the ``opposite functor'' to $F$
  (in some sense, a composition of the opposite functor $\D \to \D\op$ and $F$).

  Likewise, any functor $G : \C \to \D$
  must also act as a contravariant functor $\C\op \to \D\op$.
  If $GfGg = G(fg)$, then can apply the ``same'' mapping between objects and morphisms to get
  $Gg\op Gf\op = G(g\op f\op) = G(gf)\op$.
\end{sol}

\begin{xca}\label{xca:comma}
  Given functors $F : \D \to \C$, and $G : \E \to \C$,
  show that there is a comma category $F \comma G$ which has
  \begin{itemize}[nosep]
    \item as objects, triples $(d \in \D, e \in \E, f : Fd \to Ge \in \C)$, and
    \item as morphisms $(d,e,f) \to (d',e',f')$, a pair of morphisms $(h : d \to d', k : e \to e')$
  \end{itemize}
  so that
  \begin{center}
    \begin{tikzcd}
      Fd \arrow[r, "f"] \arrow[d, "Fh"'] & Ge \arrow[d, "Gk"] \\
      Fd' \arrow[r, "f'"']               & Ge'
    \end{tikzcd}
  \end{center}
  commutes in $\C$, i.e., $f' (Fh) = (Gk) f$.
  Define a pair of projection functors $\dom : F \comma G \to \D$
  and $\cod : F \comma G \to \E$.
\end{xca}
% TODO feels very tedious

\begin{xca}
  Define functors to construct the slice categories $c/\C$ and $\C/c$ of \cref{xca:slice}
  as special cases of comma categories constructed in \cref{xca:comma}.
  What are the projection functors?
\end{xca}
\begin{sol}
  For $c/\C$, let $\D$ be the category containing only $c$ and its identity $\id_c$.
  Then, let $F : \D \to \C$ be the ``identity'' such that $Fc = c$.
  Define $G : \C \to \C$ to also be the identity functor such that $Gx = x$ and $Gf = f$.

  Then, objects in $F \comma G$ are triples $(c, x \in \C, f : c \to x \in \C)$
  and morphisms $(c,x,f) \to (c,y,g)$ are pairs $(\id_c : c \to c, h : x \to y)$
  such that the diagram
  \begin{center}
    \begin{tikzcd}[column sep=small]
                    & c \dlar["f"'] \drar["g"] \arrow[loop, distance=2em, in=125, out=55, "\id_c"'] &   \\
      x \ar[rr,"h"] &                                                                               & y
    \end{tikzcd}
  \end{center}
  commutes.
  This is almost exactly the definition of $c/\C$.
  The domain functor sends every object $(c,x,f)$ to $c$ and morphism to $\id_c$.
  The codomain functor sends every object $(c,x,f)$ to $x$
  and morphism to $(\id_c,h)$ to $h$,
  which recovers the exact definition of $c/\C$.

  For $\C/c$, consider $G \comma F$, whose objects are triples $(x \in \C, c, f : x \to c \in \C)$
  and morphisms $(x,c,f) \to (y,c,g)$ are pairs $(h : x \to y, \id_c : c \to c)$ such that
  \begin{center}
    \begin{tikzcd}[column sep=small]
      x \drar["f"'] \ar[rr,"h"] &                                                         & y \dlar["g"] \\
                                & c \arrow["\id_c"', loop, distance=2em, in=305, out=235] &
    \end{tikzcd}
  \end{center}
  commutes.
  This is almost exactly the definition of $\C/c$.
  This time, the domain functor sends $(x,c,f) \mapsto x$ and $(h,\id_c) \mapsto h$,
  which gives us exactly the definition of $\C/c$.
  The codomain functor sends $(x,c,f) \mapsto c$ and $(h,\id_c) \to \id_c$.
\end{sol}

\begin{xca}
  Lemma 1.3.8 shows that functors preserve isomorphisms.
  Find an example to demonstrate that functors need not reflect isomorphisms:
  that is, find a functor $F : \C \to \D$ and a morphism $f$ in $\C$
  so that $Ff$ is an isomorphism in $\D$ but $f$ is not an isomorphism in $\C$.
\end{xca}
\begin{sol}
  Imagine the two categories $\C$ (\begin{tikzcd}x \rar["f"] & y\end{tikzcd})
  and $\D$ (\begin{tikzcd}a \rar["g", bend left] & b \lar["h", bend left]\end{tikzcd})
  such that $g^{-1} = h$.

  Define $F : \C \to \D$ such that $Fx = a$, $Fb = b$, and $Ff = g$.

  Then, $Ff = g$ is an isomorphism but $f$ is not.
\end{sol}

\begin{xca}
  For any group $G$, we may define other groups:
  \begin{itemize}[nosep]
    \item the center $Z(G) = \{ h \in G \mid \forall g \in G, hg = gh \}$,
    \item the commutator subgroup $C(G)$, the subgroup generated by elements $ghg^{-1}h^{-1}$, and1
    \item the automorphism group $\Aut(G)$, the group of isomorphisms $\phi : G \to G$ in $\cat{Group}$.
  \end{itemize}
  Trivially, all three constructions define a functor from the discrete category of groups
  (with only identity morphisms) to $\cat{Group}$.
  Are these constructions functorial in
  \begin{itemize}[nosep]
    \item the isomorphisms of groups? That is, do they extend to functors $\cat{Group_{iso}} → \cat{Group}$?
    \item the epimorphisms of groups? That is, do they extend to functors $\cat{Group_{epi}} → \cat{Group}$?
    \item all homomorphisms of groups? That is, do they extend to functors $\cat{Group} → \cat{Group}$?
  \end{itemize}
\end{xca}
% TODO: requires group theory

\begin{xca}
  Show that the construction of the set of conjugacy classes of elements of a group is functorial,
  defining a functor $\fn{Conj} : \cat{Group} \to \cat{Set}$.
  Conclude that any pair of groups whose sets of conjugacy classes of elements
  have differing cardinalities cannot be isomorphic.
\end{xca}
% TODO: requires group theory

\end{document}