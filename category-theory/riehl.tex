\documentclass[notes,tikz]{agony}
\usetikzlibrary{cd}
\title{\emph{Category Theory in Context} Exercises}
\renewcommand{\thexca}{\thesection.\roman{theorem}}
\newcommand{\inenum}{\vspace{-1.8em}}
\setlist[enumerate,1]{(i)}

\begin{document}
\thispagestyle{firstpage}
\renewcommand{\contentsname}{Exercises\\{\Large from Riehl, \emph{Category Theory in Context}}}
\tableofcontents

I am self-studying this alongside Aluffi, \emph{Algebra:\ Chapter 0},
so there is a bit of mixed notation.

\chapter{Categories, Functors, Natural Transformations}

\section{Abstract and concrete categories}

\begin{xca}
  \leavevmode
  \begin{enumerate}
    \item Consider a morphism $f : x \to y$.
          Show that if there exists a pair of morphisms $g, h : y \toto x$
          so that $g f = 1_x$ and $f h = 1_y$, then $g = h$ and $f$ is an isomorphism.
    \item Show that a morphism can have at most one inverse isomorphism.
  \end{enumerate}
\end{xca}
\begin{enumerate}
  \item \begin{prf}\inenum
          We can compose together $gfh = (gf)h = 1_Xh = h$.
          But $gfh = g(fh) = g1_y = g$ by associativity.
          Therefore, $g = h$.
          Since $gf = 1_x$ and $fh = fg = 1_y$, $f$ is an isomorphism.
        \end{prf}
  \item \begin{prf}\inenum
          Suppose $f : x \to y$ has two inverses $g$ and $h$.
          Then, as above, $g = g1_y = g(fh) = (gf)h = 1_xf = h$ by associativity.
        \end{prf}
\end{enumerate}

\begin{xca}
  Let $\cat C$ be a category.
  Show that the collection of isomorphisms in $\cat C$ defines a subcategory,
  the maximal groupoid inside $\cat C$.
\end{xca}
\begin{prf}
  Let $X$, $Y$, and $Z$ be objects in $\cat C$.
  We must show identity and composition, since we get associativity for free from $\cat C$.

  The identity morphisms $1_X$ are isomorphisms, since they are their own inverses.
  Therefore, they are present in the maximal groupoid.

  Now, if $f : X \to Y$ and $g : Y \to Z$ are isomorphisms,
  then the composition $gf : X \to Z$ is an isomorphism with inverse $f^{-1}g^{-1}$
  since $gff^{-1}g^{-1} = gg^{-1} = 1_Z$ and $f^{-1}g^{-1}gf = f^{-1}f = 1_X$.

  Therefore, the maximal groupoid is in fact a category.
\end{prf}

\begin{xca}
  For any category $\cat C$ and any object $c \in \cat C$, show that:
  \begin{enumerate}
    \item There is a category $c/\cat C$ whose objects are morphisms
          $f : c \to x$ with domain $c$ and
          in which a morphism from $f : c \to x$ to $g : c \to y$
          is a map $h : x \to y$ between the codomains so that the triangle
          \begin{center}
            \begin{tikzcd}[column sep=small]
                            & c \dlar["f"] \drar["g"] &   \\
              x \ar[rr,"h"] &                         & y
            \end{tikzcd}
          \end{center}
          commutes, i.e., so that $g = hf$.
    \item There is a category $\cat C/c$ whose objects are morphisms
          $f : x \to c$ with codomain $c$ and
          in which a morphism from $f : x \to c$ to $g : y \to c$
          is a map $h : x \to y$ between the codomains so that the triangle
          \begin{center}
            \begin{tikzcd}[column sep=small]
              x \ar[rr,"h"]\drar["f"] &   & y \dlar["g"] \\
                                      & c &
            \end{tikzcd}
          \end{center}
          commutes, i.e., so that $f = gh$.
  \end{enumerate}
  The categories $c/\cat C$ and $\cat C/c$ are called slice categories of $\cat C$ under and over $c$, respectively.
\end{xca}
\begin{enumerate}
  \item See Exercise I.3.7, Aluffi.
  \item See Example I.3.5, Aluffi.
\end{enumerate}

\section{Duality}

\begin{xca}
  Defining $\cat C/c$ to be $(c/(\cat C\op))\op$,
  deduce Exercise 1.1.iii(ii) from Exercise 1.1.iii(i).
\end{xca}
\begin{prf}
  We must establish that $\cat C/c$ is in fact $(c/(\cat C\op))\op$.
  Then, everything (inverses, composition, associativity) follows immediately from duality.

  First, notice that in $c/(\cat C\op)$,
  we have objects that are morphisms in $\cat C\op$, i.e.,
  the same that are in $\cat C$ but backwards:
  \begin{center}
    \begin{tikzcd}[column sep=small]
                       & c \dlar["f\op" above] \drar["g\op"] &   \\
      x \ar[rr,"h\op"] &                                     & y
    \end{tikzcd}
  \end{center}
  To get to the desired commutative diagram, we have to apply the opposite operation once more.
\end{prf}

\begin{xca}
  \leavevmode
  \begin{enumerate}
    \item Show that a morphism $f : x \to y$ is a split epimorphism
          in a category $\cat C$ if and only if for all $c \in \cat C$,
          the post-composition function $f_* : \cat C(c,x) \to \cat C(c,y)$ is surjective.
    \item Argue by duality that $f$ is a split monomorphism if and only if
          for all $c \in \cat C$, the pre-composition function
          $f^* : \cat C(y,c) \to \cat C(x,c)$ is surjective.
  \end{enumerate}
\end{xca}
\begin{enumerate}
  \item \begin{prf}\inenum
          ($\Rarr$) Suppose $f$ is a split epimorphism, i.e.,
          it is a left inverse of some morphism $g : y \to x$.
          That is, $fg = 1_y$.

          Let $h : c \to y$.
          We must show $f_*(j) = fj = h$ for some $j : c \to x$.
          Notice that $f(gh) = (fg)h = 1_y h = h$.
          Therefore, if we let $j := gh$, we are done.

          ($\Larr$) Suppose $f_*$ is surjective, i.e.,
          for all $g : c \to y$ there exists an $h : c \to x$
          such that $f_*(h) = fh = g$.
          In particular, for $g' = 1_y$, there exists $h' : y \to x$
          such that $fh' = 1_y$.
          That is, $f$ is a split epimorphism, the retraction of $h'$.
        \end{prf}
  \item \begin{prf}\inenum
          Apply part (i) to the category $\cat C\op$:
          \begin{quote}
            $f\op \in \cat C\op(x,y)$ is a split epimorphism
            if and only if for all $c \in \cat C\op$,
            the post-composition function $f_*\op : \cat C\op(c,x) \to \cat C\op(c,y)$ is surjective.
          \end{quote}
          But this is exactly the same as saying
          \begin{quote}
            $f \in \cat C(y,x)$ is a split monomorphism
            if and only if for all $c \in \cat C$,
            the pre-composition function $f^* : \cat C\op(x,c) \to \cat C\op(y,c)$ is surjective.
          \end{quote}
          because $\cat C\op(x,y) = \cat C(y,x)$
          and a split epimorphism $f\op g\op = 1_y$
          becomes a split monomorphism $gf = 1_y$.
        \end{prf}
\end{enumerate}

\begin{xca}
  Prove Lemma 1.2.11 by proving either (i) and (ii), then arguing by duality.
  Conclude that the monomorphisms in any category define a subcategory of that category
  and dually that the epimorphisms also define a subcategory.
\end{xca}
\begin{prf}
  (i) Let $f : x \mono y$ and $g : y \mono z$.
  We must show that $gf$ is monic, i.e.,
  $gfh = gfk \implies h = k$ for all $h,k : c \toto x$.
  Suppose $gfh = gfk$.
  Since $g$ is monic, we know that $g(fh) = g(fk) \implies fh = fk$.
  But then, since $f$ is monic, $fh = fk \implies h = k$.

  (ii) Let $f : x \to y$ and $g : y \to z$ such that $gf : x \mono z$.
  We must show $f$ is monic, i.e., $fh = fk \implies h = k$ for all $h,k : c \toto x$.
  Suppose $fh = fk$.
  Then, by pre-composing $g$, $(gf)h = (gf)k$.
  Since $gf$ is monic, $h = k$.

  Now, we can show that the subcategory $\cat{C_{Monic}}$ of the same objects
  as $\cat C$ and with only its monomorphisms is a category.
  First, since for all $h,k \in \cat C(c,x)$,
  $1_xh = 1_xk \implies h = k$, we have the identities in $\cat{C_{Monic}}$.
  Then, due to (i), the compositions of all morphisms are in the subcategory.
  Since the objects are unchanged, we can conclude $\cat{C_{Monic}}$ is a subcategory.

  Dually, the opposite monomorphism $f\op$ in $\cat C\op$ such that
  $f\op h\op = f\op k\op \implies h\op = k\op$ is the epimorphism
  $f$ in $\cat C$ such that $hf = kf \implies h = k$.
  Therefore, the dual of (i) is (i') and the dual of (ii) is (ii').
  Finally, the dual of the existence of the subcategory $\cat{C_{Monic}}$
  is a subcategory $\cat{C_{Epic}}$ of epimorphisms.
\end{prf}

\skipto{6} % TODO: requires ring theory
\begin{xca}
  Prove that a morphism that is both a monomorphism
  and a split epimorphism is necessarily an isomorphism.
  Argue by duality that a split monomorphism that is
  an epimorphism is also an isomorphism.
\end{xca}
\begin{prf}
  Suppose $f : x \mono y$ is also a split epimorphism such that $fg = 1_y$ for some $g : y \to x$.
  Then, since $f$ is monic, $f(gf) = (fg)f = (1_y)f = f = f(1_x) \implies gf = 1_x$.
  That is, $f$ is an isomorphism.

  By duality, a split epimorphism $f\op g\op = 1_y$ is a split homomorphism $gf = 1_y$.
  Therefore, a split homomorphism that is an epimorphism is an isomorphism too.
\end{prf}

\begin{xca}
  Regarding a poset $(\cat P, \leq)$ as a category,
  define the supremum of a subcollection of objects $a \in \cat P$
  in such a way that the dual statement defines the infimum.
  Prove that the supremum of a subset of objects is unique, whenever it exists,
  in such a way that the dual proof demonstrates the uniqueness of the infimum.
\end{xca}
\begin{prf}
  Let $a$ be the object such that $f_x : x \to a$ exists for all other $x$ in the subcollection.
  That is, for every object $x$, $y \leq a$.
  Then, the dual definition is the object $a\op$ in $\cat P\op$
  such that $f\op_X : x \to a$ always exists, i.e.,
  the object $a'$ in $\cat P$ such that $f_x : a \to x$ always exists.

  Suppose there are two suprema $a$ and $a'$.
  Then, by the definition for $a$, we just have a morphism $a \to a'$.
  But if $a'$ is a supremum, there must also be a morphism $a' \to a$.
  Therefore, $a \leq a' \leq a$, which means $a = a'$ since $\cat P$ is a poset
  (and not a pre-ordered set).
\end{prf}

\section{Functoriality}

\begin{xca}
  What is a functor between groups, regarded as one-object categories?
\end{xca}
\begin{sol}
  A functor $F : \cat G \to \cat H$ maps the single object
  $FG = H$ and each morphism (i.e., group element)
  such that for all $a,b : G \toto G$, $F(ab) = (Fa)(Fb)$ and $Fe_G = e_H$.

  That is, in the language of groups, a functor is a group homomorphism.
\end{sol}

\begin{xca}
  What is a functor between preorders, regarded as categories?
\end{xca}
\begin{sol}
  A functor $F : (\cat P, \leq) \to (\cat Q, \preccurlyeq)$
  sends the objects of $\cat P$ to objects in $\cat Q$
  such that if $a \leq b$, then $Fa \preccurlyeq Fb$.
  That is, we can regard $F$ as a set function $P \to Q$
  that is increasing with respect to the respective preorders.
\end{sol}

\begin{xca}
  Find an example to show that the objects and morphisms in the image of a functor
  $F : \cat C \to \cat D$ do not necessarily define a subcategory of $\cat D$.
\end{xca}
\begin{sol}
  Let $\cat C$ be a groupoid with two groups
  generated by two elements $A = \ev{a}$ and $B = \ev{b}$:
  \begin{center}
    \begin{tikzcd}
      A \arrow["a"', loop, distance=2em, in=305, out=235] & B \arrow["b"', loop, distance=2em, in=305, out=235]
    \end{tikzcd}
  \end{center}
  Let $\cat D$ be a groupoid with one group $G = \ev{\alpha,\beta}$.
  Now, let $F : \cat C \to \cat D$ with $FA = FB = G$,
  $Fa = \alpha$, and $Fb = \beta$.

  The image of $F$ is not a category.
  Both $\alpha$ and $\beta$ are in the image, but their composition $\beta\alpha$ is not
  since there was no $ba$ composition in $\cat C$.
\end{sol}

\begin{xca}
  Verify that the constructions introduced in Definition 1.3.11
  (functors $\cat C(c,-)$ and $\cat C(-,c)$ represented by $c$) are functorial.
\end{xca}
\begin{prf}
  Recall that $\cat C(c,-) : \cat C \to \cat{Set}$
  sends $x$ to the set $\cat C(c,x)$
  and $f : x \to y$ to the post-composition function
  $f_* : \cat C(c,x) \to \cat C(c,y) : h \mapsto fh$.
  Verify the functoriality axioms:

  Let $f : x \to y$ and $g : y \to z$ be composable morphisms where $gf : x \to z$.
  Then, $(gf)_* : \cat C(c,x) \to \cat C(c,z)$.
  We can verify that $(gf)_*(h) = (gf)h = g(f(h)) = g(f_*h) = g_*f_*h$ for all $h : c \to z$.
  That is, $(gf)_* = g_*f_*$.

  Consider an identity $1_x : x \to x$.
  Then, the post-composition function $(1_x)_* : \cat C(c,x) \to \cat C(c,x)$
  sends morphisms to themselves because $h1_x = h$ and $1_xg = g$ by definition,
  i.e., it is $1_{\cat C(c,x)}$.
\end{prf}

\begin{xca}
  What is the difference between a functor $\cat C\op \to \cat D$
  and a functor $\cat C \to \cat D\op$?
  What is the difference between a functor $\cat C \to \cat D$
  and a functor $\cat C\op \to \cat D\op$?
\end{xca}
\begin{sol}
  If $F : \cat C\op \to \cat D$ sends $f\op : a \to b$ to $g : x \to y$
\end{sol}

\end{document}